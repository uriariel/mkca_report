Multi Key Collision Attacks is an attack where we find a ciphertext - C that decrypts under k different keys.

In practice, to perform a multi key collision attack against AES-GCM
we need to solve a system of linear equations.
This is possible because of the algebraic properties of the universal hashing in AES-GCM.

We implemented a solver for such a system of linear equations for 2 keys, 3 keys, and an arbitrary constraints.

\hfill \break

\begin{tabular}{ |p{3cm}||p{3cm}|p{3cm}|p{5cm}| }
    \hline
    Key count & Time       & Ciphertext size & Ciphertext size/Key count \\
    \hline
    2         & 0.04199    & 4               & 2                         \\
    4         & 0.0934324  & 6               & 1.5                       \\
    8         & 0.0628430  & 10              & 1.25                      \\
    16        & 0.1656253  & 18              & 1.125                     \\
    64        & 0.708905   & 66              & 1.03125                   \\
    256       & 7.526772   & 258             & 1.0078125                 \\
    1024      & 46.03521   & 1026            & 1.001953125               \\
    4096      & 996.350043 & 4098            & 1.00048828125             \\
    \hline
\end{tabular}

\hfill \break

From the above table we can clearly see that the resulting ciphertext is linear in the size of the key space.
and in fact are exactly the key count and two additional blocks.

The time to solve this linear equations without parallel optimizations is around polynomial third degree time.
The main cryptanalytic step for our attacks is constructing (what we call) key multi-collisions, in which a single
AEAD ciphertext can be built such that decryption succeeds under some number k of keys.

We build key multi-collision ciphertexts of length $O(k)$ in $O(k^2)$ time using polynomial
interpolation from off-the-shelf libraries, making them reasonably scalable even to large k.
We can further optimize to get $O(k \log^2 k)$ running time using a different
polynomial interpolation technique\cite{BORODIN1974366}.

Given access to an oracle that reveals whether decryption succeeds, our key multi-collisions
for AES-GCM enables a partitioning oracle attack that recovers the secret key k in
roughly $m + \log k $ queries in situations where possible keys fall in a set of size $d = m \cdot k$.