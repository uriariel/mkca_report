Key commitment guarantees that a ciphertext can only be
decrypted under the same key it was encrypted with.
If we are to find a ciphertext that decrypts into two valid plaintexts under two different keys do not commit with respect to the key.

Key commitment was initially studied by Farshim et al. [FOR17] and while it might seem like an academic pursuit, Dodis et
al. [DGRW18] and Grubbs et al. [GLR17a] show how to exploit AEADs that do not commit to their key.
Nevertheless, AEAD key commitment has not received the required attention and can be overlooked in deployment.

A committing encryption scheme is a scheme for which it is computationally unfeasible to find two keys and a ciphertext that decrypts under both.
Security standards for committing AEADs were first formalized by Farshim et al. [FOR17].