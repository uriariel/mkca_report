The design of encryption historically separated the goals of confidentiality and authenticity, which led to widespread
deployment of encryption schemes vulnerable to chosen ciphertext attacks (CCA).

Subsequently, researchers showed how to exploit CCAs to recover plaintext data, most notably via padding and
format oracle attacks
As a result, cryptographers now advocate the use of authenticated encryption with associated data (AEAD) schemes and CCA-secure
public key encryption.
There has since been a shift to adopt fast CCA-secure schemes, notably AES-GCM, and XSalsa20/Poly1305.

Such schemes do not target being committing.
Thus far, key commitment has not been considered an essential security goal for most cryptographic applications.
This is perhaps because attacks exploiting key collision have arisen in relatively niche applications like
message franking for encrypted messages.

Grubbs et al. introduce partitioning oracle attacks, a new type of CCA in password-authenticated key exchange (PAKE).
A partitioning oracle arises when an attacker can:
\begin{enumerate}
    \item efficiently prepare ciphertexts that will decrypt under a large number of keys
    \item submit those ciphertexts to an oracle that tells whether decryption succeeds or fails.
\end{enumerate}

This enables gaining information about the password.