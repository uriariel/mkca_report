\subsection*{Authenticated Encryption}\label{subsec:authenticated-encryption}
Authenticated Encryption (AE)\cite{rfc5116} is a well-studied primitive which avoids the pitfalls of unauthenticated
Symmetric-key encryption with relatively small performance overhead
AE schemes are used in widely adopted protocols and are the default Symmetric option in modern cryptographic libraries.

The design of encryption historically separated the goals of confidentiality and authenticity, which led to widespread
deployment of encryption schemes vulnerable to chosen ciphertext attacks (CCA).

Subsequently, researchers showed how to exploit CCAs to recover plaintext data, most notably via padding and
format oracle attacks
As a result, cryptographers now advocate the use of authenticated encryption with associated data (AEAD) schemes and CCA-secure
public key encryption.

There has since been a shift to adopt fast CCA-secure schemes, notably AES-GCM, and XSalsa20/Poly1305.

\subsection*{AES-GCM}\label{subsec:aes-gcm}
AES-GCM is an AEAD scheme that composes AES in counter mode with a special MAC
that uses an XOR-universal hash function called GHASH.
Encryption takes in a nonce N, an AES key K, associated data AD, and plaintext $M_{1},\ldots,M_{n}$ without loss of generality (AES-GCM specification handles various length).
and it outputs a ciphertext $C_{1},\ldots,C_{n}$ and an authentication tag T.
The ciphertext blocks $C_{1},\ldots,C_{n}$ are generated using counter mode with E, and the tag T
is computed by applying GHASH to AD and $C_{1},\ldots,C_{n}$.
Decryption also computes the tag, compares it with T, and, if successful, outputs the counter-mode decryption of the ciphertext
blocks.
For further detailes please revise the original papers.\cite{mg04}

\subsection*{GHASH}\label{subsec:ghash}
We will omit associated data for simplicity.

For a key K, GHASH first derives a hash key $H = E_k(0^n)$.
It then hashes by computing
$h = C_1 \cdot H^{m+1} \oplus\cdot\cdot\cdot\oplus C_{m-1} \cdot H^3 \oplus C_m^\ast \cdot H^2 \oplus L \cdot H$
where $C_m^\ast$ is $C_m$ concatenated with enough zeros to get an
n-bit string and L is an n-bit encoding of the length of the
message.
The maximum plaintext length is $2^{39} - 256$.
The multiplications are performed over the finite field GF($2^{128}$) with a particular fixed irreducible polynomial.